\section{Einleitung}
\label{sec:einleitung}
Die Programmierung eines PIC-Microchip unter Linux hat sich für mich als Linux-Nutzer damals zu einer kleinen Herausforderung entwickelt, weswegen ich versucht habe eine möglichst nahe Programmierung über den Terminal zu suchen. Diese Methode hat den Vorteil, dass man mit einem Editor seiner Wahl Assembler-Code schreiben kann und ihn einfach über Commands im Terminal ausführen kann.

In diesen kleinen Anleitung erkläre ich, wie man dieses Verfahren unter Linux (Ubuntu) aufsetzen lässt. In einem optionalen Schritt zeige ich noch weiterhin, wie man über Visual Studio Code seine Assembler Programmierung aufsetzen kann.

Bei Fragen stehe ich Ihnen gerne über \href{https://github.com/ManeLippert/AssemblerVSCode}{GitHub} zur Verfügung. Schreiben sie einfach ein Issue zu Ihrem Problem.